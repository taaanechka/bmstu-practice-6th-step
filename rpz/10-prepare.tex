\chapter{Подготовка к проведению конкурса}

\section{Установка необходимого ПО}

В процессе подготовки аудитории к проведению салона <<Шаг в будущее>> потребовалось установить необходимое ПО. 
Для этого был произведен обзвон каждого из участников данного конкурса. В результате для участников, нуждающихся 
в стационарном компьютере, было установлено необходимое для демонстрации работы их проектов ПО.

После обзвона участников был определен следующий список необходимых для их проектов языков программирования.
\begin{enumerate}
    \item Python3.
    \item C++.
    \item C\#
    \item .NET framework.
    \item Pascal ABC.NET.
\end{enumerate}

Стационарные компьютеры были проверены на работоспособность. После чего на компьютерах, находящихся в рабочем состоянии, 
было установлено необходимое ПО. С установкой ПО сложностей не возникло. Стационарных компьютеров, находящихся в рабочем состоянии, 
хватило всем желающим.

Некоторым участникам для демонстрации работы их проектов был необходим интернет. Это также выяснилось в процессе обзвона участников 
до проведения конкурса. В том случае, если участнику требовался интернет, ему было предложено авторизироваться через бауманскую сеть. 
В случае, если участник конкурса заранее предоставил информацию о необходимости использования интернета, ему был выдан доступ под его 
собственной учетной записью. В случае, если это выяснялось на месте, ему был предоставлен доступ к сети с использованием учетной записи 
одного из организаторов со стороны студенческого жюри. После проведения конкурса был осуществлен выход из таких учетных записей. 

Многие участники конкурса принесли собственные электронные устройства с установленными необходимыми программами.

%Таким образом, установка необходимого ПО в процессе подготовки к проведению салона <<Шаг в будущее>> прошла успешно.

%\section{Составление схемы рассадки участников}
%
%По результатам проведенного обзвона была получена информация, необходимая для рассадки участников конкурса в аудитории. В зависимости от того, 
%требовался ли конкретному участнику стационарный компьютер, было определено его место в схеме рассадки. Стационарные компьютеры были проверены на 
%работоспособность.
%
%План рассадки участников конкурса представлен на рисунке~\ref{img:sh}.
%
%\imgw{sh}{ht!}{0.9\textwidth}{Схема рассадки участников конкурса}

%Также были подготовлены именные указатели для каждого из участников данного конкурса. 
%Это существенно облегчило процесс нахождения ими своего места перед проведением конкурса.
%Пример такого указателя приведен на рисунке~\ref{img:example}.
%
%\imgw{example}{ht!}{0.8\textwidth}{Пример бумажного указателя}

%\clearpage
%\section{Подготовка именных бейджей}
%
%Для возможности идентифицировать организаторов во время проведения конкурса были заранее подготовлены бейджи. 
%Также были подготовлены карточки и с информацией, позволяющей идентифициоровать организаторов, состав которых представлен в соответсвующем документе \cite{step4}. Бейджи крепились на одежду.
%
%Бейжди были закуплены на всех организаторов. На карточках были указаны название ВУЗа, в котором проводится программный салон, 
%ФИО организатора конкурса и его должность.

%\section{Составление студенческих рецензий}
%
%Работы участников были предоставлены частично в электронном, частично в бумажном виде. Все работы были изучены 
%и оценены членами студенческого жюри. На каждую из работ участников была составлена рецензия. 
%
%Для составления рецензии на каждую из работ членам студенческого жюри были выделены специальные бланки. 
%Каждый бланк состоял из двух частей -- оценки работы и резюме рецензента. В оценочной части необходимо поставить оценку 
%от 0 до 3 по пяти критериям:
%
%\begin{itemize}
%    \item структура и оформление работы (качество оформления, грамотность содержания, ошибки, опечатки, выводы);
%    \item логика изложения, оригинальность мышления, творческий подход;
%    \item используемые методы (причины использования данных методов, эффективность, точность и простота методов);
%    \item оригинальность тематики проекта;
%    \item научное и практическое значение работы.
%\end{itemize}
%
%В первой части работа оценивается по 5 вышеприведенным критериям, сумма всех баллов не может превышать 15. 
%
%Во второй части члену студенческого жюри необходимо сформулировать резюме, а также написать возникшие в 
%ходе прочтения отчета вопросы, высказать свои замечания и обозначить недостатки работы.
%
%Для составления рецензий были предоставлены бланки, образец которых приведен в приложении .

%\section{Обеспечение порядка проведения салона}
%
%Была проведена подготовка аудитории к проведению программного салона: уборка, размещение бумажных указателей на столах. 
%Также некоторые участники сочли необходимым проверить работу их проектов на предустановленном ПО на стационарных компьютерах 
%до проведения конкурса. Для этого было отведено специальное время, когда требуемая аудитория была свободна.
%Организаторами конкурса были использованы бейджи. 
%
%Участникм конкурса была оказана помощь в сопровождении их до места проведения салона, 
%проведен инструктаж о правилах поведения в аудитории. Оргкомитет обязан ознакомить всех участников конкурса с требованиями безопасности, 
%которые они должны соблюдать во время проведения программного салона.

%\section{Оценка работ участников}
%
%Для дальнейшей оценки студенческого жюри необходимо внимательно изучить работы участников конкурса и составить студенческие рецензии на них. 
%Для этого все студенты-организаторы разбиваются на группы по 2 человека, чтобы объективно оценить всех абитуриентов.
%
%Сначала происходит ознакомление с докладом автора, опрос по теме выступления. 
%Далее необходимо оценить работу по следующее:
%\begin{itemize}
%    \item структуру и оформление работы;
%    \item актуальность тематики работы;
%    \item полноту раскрытия темы;
%    \item логику изложения, оригинальность мышления;
%    \item используемые методы и обоснование их использования;
%    \item наличие в тексте работы заимствований из источников, в том числе из ресурсов сети Интернет;
%    \item наличие предложений по практическому использованию программы;
%    \item вклад автора в выбранную тему.
%\end{itemize}
%
%Также оргкомитетом учитываются и другие критерии:
%\begin{itemize}
%    \item грамотность, полнота и четкость изложения проблемы;
%    \item качество доклада, защиты и умение ориентироваться в теме;
%    \item актуальность решаемой проблемы, новизна и достоверность результатов;
%    \item использование современных методов решения проблемы;
%    \item использование знаний внешкольной программы;
%    \item научное и практическое значение работы;
%    \item творческая составляющая в подходе, процессе и защите работы.
%\end{itemize}
%
%После прослушивания каждому участнику были заданы соответствующие вопросы по теме работы. Члены жюри выставили оценки по вышеприведенным критериям.
%
%Работая в группе с Алферовой Ириной, мы просмотрели и оценили 9
%работ школьников.
%// todo: добавить фото
%
%Была составлена результирующая таблица с оценками участников, со стороны как студенческого, так и преподавательского жюри. После подсчета набранных участниками конкурса баллов, были определены три победителя этапа олимпиады «Шаг в будущее>> и три участника, выбранных студенческим жюри.
%Призерам была вручена научная литература, выпущенная в издательстве МГТУ им. Н.Э. Баумана.