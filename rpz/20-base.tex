\chapter{Основная часть}

\section{Работа участника}

\subsection{Сведения об участнике и его работе}

Студенческому жюри были представлены работы участников конкурса, после чего была произведена их оценка. Далее будет представлен анализ одной из работ.

Рассматриваемая далее работа была выполнена Сиденко Олегом Генадьевичем, учеником 10-Б класса ГБОУ Школы №1571. Тема работы, представленной им на конкурсе: <<Веб-приложение для поиска фильма по цитате>>. Титульный лист этой работы представлен на рисунке~\ref{img:title-sid}.

\subsection{Цель работы}

Целью рассматриваемой работы участника является создание сервиса поиска фильмов по фразе из них. Данная работа способствует обобщению знаний, закрепленных на практике. В качестве перспектив развития рассматривается расширение и адаптация программного продукта внутри заинтересованных организаций.

\subsection{Описание ПО участника}

В предоставленной участником конкурса версии программного обеспечения реализован сервис, который может искать фильмы как по полной цитате, так и по ее части. При этом в цитате ликвидируются различия в регистре, неверно расставленные знаки препинания, опечатки и орфографические ошибки. Работа данного сервиса предусмотрена в двух вариантах: веб-сервис и Telegram-бот. Также предусмотрена авторизация пользователя и добавление понравившегося фильма в избранное.

Полученное в результате работы ПО может использоваться как часть веб-сервисов по подбору фильмов, например как Кинопоиск.

\subsection{Поставленные задачи}

Участник конкурса поставил перед собой следующие задачи:
\begin{itemize}
	\item создать базу данных фильмов и их текстов;
	\item разработать алгоритм предобработки запроса пользователя и поиска
	информации по базе данных;
	\item создать сайт, демонстрирующий работу системы.
\end{itemize}

\imgw{title-sid}{ht!}{0.8\textwidth}{Титульный лист рассматриваемой работы участника конкурса}

\subsection{Актуальность темы}

Участник утверждает, что <<поиск зачастую не затрагивает некоторые ниши, которые бывают нужны при запросах в интернет, например, поиск фильма по звучавшей в нем фразе. Это может быть полезно, когда не удается вспомнить название фильма, но Вы помните
цитату из него. Но если цитата не стала крылатым выражением или просто
обсуждаемой фразой, Вы не получите ответа на вопрос, что же это был за
фильм. Не даст Вам поиск ответа и в том случае, если цитата оборвана, отложилась в Вашей памяти немного неточно или
используется где-то кроме этого фильма, например в книге или статье>>.

Я считаю, что поставленная участником задача актуальна с учетом того, что обычные поисковики не ищут фильмы по искаженным фразам, но в свете расширения объема данных. Как человеку, способному заинтересоваться в просмотре фильма по нескольким фразам из него, мне была бы интересна версия с более полным набором данных. При тестировании работы участника находилась только часть фильмов, что связано с недостаточно полным, на мой взгляд, объемом фильмов в его базе. Сам он считает также. В дальнейшем планирует расширить базу фильмов для своего ПО.

\subsection{Формирование базы данных}

Перед непосредственным формированием базы данных требовалось определиться с типом базы.

Участник сделал выбор
в пользу реляционной базы данных, так как данные имеют четкую структуру и заранее определены. В качестве СУБД была выбрана SQLite. SQLite -- компактная встраиваемая реляционная база данных с открытым исходным кодом. Преимуществом SQLite является скорость исполнения, простота кода, встраиваемость в Python.

В базе данных представлены следующие сущности:
\begin{itemize}
	\item users -- пользователи;
	\item requests -- запросы пользователей;
	\item telegram -- телеграм-профили пользователей;
	\item liked -- понравившиеся пользователям фильмы;
	\item films -- фильмы;
	\item subtitles -- субтитры.
\end{itemize}

Схема базы данных представлена на рисунке~\ref{img:db}.
\imgw{db}{ht!}{0.8\textwidth}{Схема базы данных}

На момент презентации работы в базе содержалось 535 разных фильмов и 591274
строки с субтитрами. В телеграмм базе организован постраничный вывод
по 5 фильмов, из-за ограничений площадки: сообщения обрезаются до 10000
символов.

На рисунке~\ref{img:res1} приведен пример поиска по фразе <<Работал деньги>>. 
\imgw{res1}{ht!}{1\textwidth}{Запрос <<Работать деньги>>}

На рисунке~\ref{img:res2} приведен пример по сравнению двух похожих запросов на количество найденных фильмов. 
\imgw{res2}{ht!}{1\textwidth}{Сравнение запросов}

\subsection{Заключение}

Цель работы участника была достигнута: разработан сервис поиска фильмов по фразе из них.

Также были выполнены поставленные им задачи:
\begin{itemize}
	\item создана база данных фильмов и их текстов;
	\item разработан алгоритм предобработки запроса пользователя и поиска информации по базе данных;
	\item создан сайт, демонстрирующий работу системы.
\end{itemize}

\section{Студенческая рецензия}

Работы участников были предоставлены частично в электронном, частично в бумажном виде. Все работы были изучены 
и оценены членами студенческого жюри. На каждую из работ участников была составлена рецензия. 

Для составления рецензии на каждую из работ членам студенческого жюри были выделены специальные бланки. 
Каждый бланк состоял из двух частей -- оценки работы и резюме рецензента. В оценочной части необходимо поставить оценку 
от 0 до 3 по пяти критериям:

\begin{itemize}
    \item структура и оформление работы (качество оформления, грамотность содержания, ошибки, опечатки, выводы);
    \item логика изложения, оригинальность мышления, творческий подход;
    \item используемые методы (причины использования данных методов, эффективность, точность и простота методов);
    \item оригинальность тематики проекта;
    \item научное и практическое значение работы.
\end{itemize}

В первой части работа оценивается по 5 вышеприведенным критериям, сумма всех баллов не может превышать 15. 

Во второй части члену студенческого жюри необходимо сформулировать резюме, а также написать возникшие в 
ходе прочтения отчета вопросы, высказать свои замечания и обозначить недостатки работы.

Студенческая рецензия на работу Сиденко Олега Генадьевича приведена на рисунке~\ref{img:rewiew-sid}.
\clearpage
\imgw{rewiew-sid}{ht!}{0.8\textwidth}{Студенческая рецензия на работу Сиденко Олега Генадьевича}

%\clearpage
\section{Оценка работ участников на конкурсе}

Для дальнейшей оценки студенческого жюри необходимо внимательно изучить работы участников конкурса и заполнить таблицы с баллами. 
Для этого все студенты-организаторы разбиваются на группы по 2 человека, чтобы объективно оценить всех абитуриентов.

Сначала происходит ознакомление с докладом автора, опрос по теме выступления. 
Далее необходимо оценить работу по следующее:
\begin{itemize}
    \item структуру и оформление работы;
    \item актуальность тематики работы;
    \item полноту раскрытия темы;
    \item логику изложения, оригинальность мышления;
    \item используемые методы и обоснование их использования;
    \item наличие в тексте работы заимствований из источников, в том числе из ресурсов сети Интернет;
    \item наличие предложений по практическому использованию программы;
    \item вклад автора в выбранную тему.
\end{itemize}

Также оргкомитетом учитываются и другие критерии:
\begin{itemize}
    \item грамотность, полнота и четкость изложения проблемы;
    \item качество доклада, защиты и умение ориентироваться в теме;
    \item актуальность решаемой проблемы, новизна и достоверность результатов;
    \item использование современных методов решения проблемы;
    \item использование знаний внешкольной программы;
    \item научное и практическое значение работы;
    \item творческая составляющая в подходе, процессе и защите работы.
\end{itemize}

После прослушивания каждому участнику были заданы соответствующие вопросы по теме работы. Члены жюри выставили оценки по вышеприведенным критериям.

Работая в группе с Алферовой Ириной, мы просмотрели и оценили 9 работ абитуриентов.

Была составлена результирующая таблица с оценками участников, со стороны как студенческого, так и преподавательского жюри. После подсчета набранных участниками конкурса баллов, были определены три победителя этапа олимпиады «Шаг в будущее>> и три участника, выбранных студенческим жюри.
Призерам была вручена научная литература, выпущенная в издательстве МГТУ им. Н.Э. Баумана.